\documentclass{article}
\usepackage{hyperref}
\hypersetup{
    colorlinks=true,
    linkcolor=blue,
    filecolor=magenta,      
    urlcolor=cyan,
}
\begin{document}

\title{Titanic ML Project}
\author{Bogdanowicz Michal Kamil, Geraci Luca}

\maketitle

\begin{abstract}
The abstract text goes here.
\end{abstract}


\section{Proposition}
The proposition is using a set of machine learning methods to predict the people that would survive the Titanic sinking of 15 April 1912.

The methods that have been used are:

\begin{enumerate}  
\item Logistic Regression TODO
\item Random trees TODO
\item Neural Networks TODO
\end{enumerate}

\section{Data}
\subsection{Incomplete Data}

The Data propsed to the professor was incomplete. It heavily reduced the avialble data to perform the best practices for evaluating a model. The test and training were already separated and ground truth was missing. So the complete dataset has been taken from the internet. In addition the complete list of surviors can be found at the wikipedia page \href{https://en.wikipedia.org/wiki/Passengers_of_the_RMS_Titanic}{here} (not a ML-friendly format).
This data sadly is used to cheat on the kaggle competition. Making it a quite infamous one.

\subsection{Data Format}

\begin{itemize}
\item Ticket class
\item Survival 
\item Name
\item Sex
\item Age in years
\item sibsp : number of siblings / spouses aboard the Titanic <-- This might be problematic as it doesn't seem to make much sense.
\item parch : number of parents / children aboard the Titanic<-- This might be problematic as it doesn't seem to make much sense.
\item Ticket number	
\item Fare
\item Cabin/s assigned
\item Port of Embarkation	
\item Rescue Boat
\item Body
\item Destination
\end{itemize}

With additional notes of :
Ticket class: A proxy for socio-economic status (SES)
1st = Upper
2nd = Middle
3rd = Lower

age: Age is fractional if less than 1. If the age is estimated, is it in the form of xx.5

sibsp: The dataset defines family relations in this way
Sibling = brother, sister, stepbrother, stepsister
Spouse = husband, wife (mistresses and fiancés were ignored)

parch: The dataset defines family relations in this way
Parent = mother, father
Child = daughter, son, stepdaughter, stepson
Some children travelled only with a nanny, therefore parch=0 for them.



\subsection{Data Preparation}
The data cannot be used as it is.
The names are going to be deleted, as that should not inlfuence the resut.

\section{Logistic Regression}
This is the procedure that has been used to select the model.

\begin{enumerate}  
\item Choose the maximum polynomial to use by comparing the measures with CV (Cross Validation)

\item Check for Bias and variance by visualizing the Error in CV and Error on Training set on a graph with X as the degree of polynomial
\item Choose the regularization parameter that reducces the CV Error the most by using the polynomial choseon and Lambda going from 0.01 to 5.
\end{enumerate}



\section{Conclusion}
Write your conclusion here.

\end{document}
